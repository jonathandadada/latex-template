\documentclass[12pt]{report}

\usepackage[utf8]{inputenc}
\usepackage[T1]{fontenc} % TODO : ce renseigner sur l'encodage
\usepackage[french]{babel}
\usepackage[left=2.5cm,right=2.5cm,bottom=2.5cm,top=2.5cm]{geometry} %pour les marges
\usepackage{listings} %lstinputlisting
\usepackage{xcolor}
\usepackage{amsmath} %gather
\usepackage{graphicx}
% fancy
\usepackage{lastpage}
\usepackage{fancybox,ctable} % pour table avec lignes et colonnes
\usepackage{fancyhdr}
% end fancy
\usepackage{titlesec} % hauteur des chapitres titlespacing ne marche pas ?

%pdf cliquable
\usepackage{hyperref}
\hypersetup{
    colorlinks=true, % lien colore si == true
    linktoc=all,     % les sections et sous section sont cliquables
    linkcolor=black, % les liens sont de couleur noir
}

\titleformat{\chapter}[hang]{\bf\huge}{\thechapter}{2pc}{} % voir si je change la taille des chapitres a nouveau

% code, coloration par default
\lstset{
	%frame=tb, %bordure haute et basse
	tabsize=4,
	%numbers=left,
	breaklines=true,
	commentstyle=\color{green},
	keywordstyle=\color{purple},
	stringstyle=\color{blue},
	caption=\lstname,
	basicstyle=\small
}

\titlespacing{\chapter}{0cm}{0cm}{0.3cm}

\makeatletter
\let\ps@plain=\ps@fancy % sert à remplacer le plain en fancy
\makeatother

\title{\textbf{Rapport NUMEROTP NOMMATIERE\\
INTITULEMATIERE}}
\author{
	NOMAUTEURDOC, PRENOMAUTEURDOC\\
	{\small {ADRESSEEMAILDOC}}
}
\date{\today}

%%%%% haut de page + pied de page %%%%
\renewcommand{\headrulewidth}{1pt}
\pagestyle{fancy}
\fancyhead[L]{NOMMATIERE-Rapport-NUMEROTP}
\fancyhead[R]{\includegraphics[width=1cm]{images/urca.jpg}}

\renewcommand{\footrulewidth}{1pt}
\fancyfoot[L]{\leftmark}
\fancyfoot[C]{} 
\fancyfoot[R]{page \thepage/\pageref{LastPage}}
%%% fin haut de page + pied de page %%%

\begin{document}
	
	\begin{titlepage}

	\centering
	
	\scshape
	
	\vspace*{\baselineskip}

	\rule{\textwidth}{2pt}\vspace*{-\baselineskip}\vspace*{2pt} 
	\rule{\textwidth}{0.5pt}
	
	\vspace{0.75\baselineskip}
	
	{\LARGE Rapport NOMMATIERE\\ INTITULEMATIERE\\}
	
	\vspace{0.75\baselineskip}
	
	\rule{\textwidth}{0.5pt}\vspace*{-\baselineskip}\vspace{3pt} 
	\rule{\textwidth}{2pt} 
	
	\vspace{1.5\baselineskip} 
	
	Rapport récapitulatif du tp courant d'info0605
	
	\vspace*{3\baselineskip}
	
	Créé par :
	
	\vspace{0.5\baselineskip}
	{\scshape\Large PRENOMAUTEURDOC NOMAUTEURDOC}
	
	\vspace{0.5\baselineskip}
	
	\textit{Ufr Sciences Exactes Et Naturelles \\ URCA }
	
	\vfill
	
	\includegraphics[width=10cm]{images/urca.jpg}
	
	\vspace{0.5\baselineskip}
	
	\textbf{{\large \the\year}}

	\end{titlepage}
	
	\maketitle
	\selectlanguage{french}		
	
	%ne pas oublier : generation automatique en 2 passes (flemme)
	\tableofcontents

	\chapter{chapitre de test}

	test
	
	\chapter{démo code inclus en C}

	\section{la section ci dessous est un exemple}
	
	Je montre seulement les inclusions, vous pouvez appliquer le même principe pour chaque langage de programmation.

	\section{inclusion du code de la cellule}	
	
	\subsection{headers d'une cellule}
	\lstinputlisting[language=c]{codes/liste_chainee/cellule.h}	

	\newpage	
	
	\subsection{code d'une cellule}
	\lstinputlisting[language=c]{codes/liste_chainee/cellule.c}	
		
	\newpage		
	
	\section{inclusion du code du graphe}
	
	\subsection{headers d'un graphe}
	\lstinputlisting[language=c]{codes/graphe/graphe.h}
	
	\newpage	
	
	\subsection{code d'un graphe}
	\lstinputlisting[language=c]{codes/graphe/graphe.c}	
		
	
\end{document}